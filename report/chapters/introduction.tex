\section{Úvod}
Tato zpráva slouží jako dokumentace semestrálního projektu pro předmět NI--VMM. Zadáním projektu byla tvorba webové aplikace pro vyhledávání hudebních skladeb ve formátu MIDI na základě podobnosti melodií.

V této zprávě nejdříve blíže popíši vlastnosti projektu. Dále prozkoumám způsoby extrakce melodií a jejich porovnávání. V implementační části se zaměřím na zvolení vhodných technologií, práci s MIDI soubory (tím pádem také stručně popíši standard MIDI) a proces tvorby jak backendové, tak frontendové části aplikace. Předvedu příklady výstupů hotové aplikace a provedu základní měření rychlosti vyhledávání. Na závěr zhodnotím dosažené výsledky a navrhnu možná vylepšení aplikace.

\pagebreak
\section{Popis projektu}
Zaměřil jsem se na klasické klavírní skladby -- díky tomu, že se ve skladbě hraje jenom na jeden nástroj, je určení hlavní melodie značně jednodušší. Kromě toho jsou MIDI soubory pro klasickou hudbu na internetu velmi dostupné.

Vstupem vyhledávání je kratší soubor ve formátu MIDI (cca 4--40 po sobě jdoucích not), který obsahuje nějakou melodii. Aplikace se následně tuto melodii pokusí najít ve všech skladbách v databázi. Výstupem je seznam skladeb, které obsahují melodii nejvíce se podobající vstupní melodii. Tento seznam je setříděn od nejlepších (nejvíce podobných) výsledků po nejhorší.

Databáze obsahuje 295 klasických skladeb volně dostupných ze stránky \url{http://www.piano-midi.de/}.

Implementace projektu spočívala v několika krocích:
\begin{itemize}
    \item Extrakce melodie z MIDI souborů a datová reprezentace melodie v databázi
    \item Porovnávání dvou melodií (určení jejich podobnosti)
    \item Vyhledávání nejpodobnější melodie v celé databázi
    \item Uživatelské rozhraní přizpůsobené pro práci s hudbou (např. umožňuje přehrávání, vizualizaci melodií\ldots)
\end{itemize}
