\section{Úvod}
Tato zpráva slouží jako dokumentace semestrálního projektu pro předmět NI--VMM na téma Podobnost melodií (MIDI).

Zadáním projektu byla tvorba webové aplikace pro vyhledávání hudebních skladeb ve formátu MIDI na základě podobnosti melodií. Vstupem hledání je soubor ve formátu MIDI, který se následně porovná s extrahovanými melodiemi v databázi a pomocí různých metrik aplikace vyhledá podobné výsledky.

V této zprávě nejdříve blíže popíši vlastnosti projektu. Dále prozkoumám způsoby extrakce melodií a jejich porovnávání. V implementační části se zaměřím na zvolení vhodných technologií, práci s MIDI soubory (tím pádem také stručně popíši standard MIDI) a proces tvorby jak backendové, tak frontendové části aplikace. Předvedu příklady výstupů hotové aplikace a provedu základní měření rychlosti vyhledávání. Na závěr zhodnotím dosažené výsledky a navrhnu možná vylepšení aplikace.

\section{Popis projektu}